\documentclass[twoside]{article}

\usepackage{aistats2021}
\usepackage{amssymb}

% If your paper is accepted, change the options for the package
% aistats2021 as follows:
%
%\usepackage[accepted]{aistats2021}
%
% This option will print headings for the title of your paper and
% headings for the authors names, plus a copyright note at the end of
% the first column of the first page.

% If you set papersize explicitly, activate the following three lines:
%\special{papersize = 8.5in, 11in}
%\setlength{\pdfpageheight}{11in}
%\setlength{\pdfpagewidth}{8.5in}

% If you use natbib package, activate the following three lines:
%\usepackage[round]{natbib}
%\renewcommand{\bibname}{References}
%\renewcommand{\bibsection}{\subsubsection*{\bibname}}

% If you use BibTeX in apalike style, activate the following line:
%\bibliographystyle{apalike}

\begin{document}

% If your paper is accepted and the title of your paper is very long,
% the style will print as headings an error message. Use the following
% command to supply a shorter title of your paper so that it can be
% used as headings.
%
%\runningtitle{I use this title instead because the last one was very long}

% If your paper is accepted and the number of authors is large, the
% style will print as headings an error message. Use the following
% command to supply a shorter version of the authors names so that
% they can be used as headings (for example, use only the surnames)
%
%\runningauthor{Surname 1, Surname 2, Surname 3, ...., Surname n}

% Supplementary material: To improve readability, you must use a single-column format for the supplementary material.
\onecolumn
\aistatstitle{Instructions for Paper Submissions to AISTATS 2021: \\
Supplementary Materials}

\section{FORMATTING INSTRUCTIONS}

To prepare a supplementary pdf file, we ask the authors to use \texttt{aistats2021.sty} as a style file and to follow the same formatting instructions as in the main paper.
The only difference is that the supplementary material must be in a \emph{single-column} format.
You can use \texttt{supplement.tex} in our starter pack as a starting point, or append the supplementary content to the main paper and split the final PDF into two separate files.

Note that reviewers are under no obligation to examine your supplementary material.

\section{Derivations of density corrections}

\subsection{Simplex augmented softmax parameterization}

We define the transformation $\phi: \mathbb{R} \backslash \{0\} \to \Delta^{n-1} \times \mathbb{R}_{>0}: z \mapsto (x, r)$,
where $r = \sum_{i=1}^n e^{z_i}$ and $x_i = \frac{1}{r} e^{z_i}$.

First we compute the scalar derivatives:

$$
\begin{aligned}
    \frac{\mathrm{d} r}{\mathrm{d} z_j} &= e^{z_j} = r x_j\\
    \frac{\mathrm{d} x_i}{\mathrm{d} z_j} &= \delta_{ij} \frac{1}{r} e^{z_i} - \frac{1}{r^2} e^{z_i} \frac{\mathrm{d} r}{\mathrm{d} z_j} = \delta_{ij} x_i - x_i x_j,
\end{aligned}
$$

where $\delta_{ij} = \begin{cases} 1 &\text{ if } i = j \\ 0 &\text{otherwise}\end{cases}$ is the Kronecker delta.

If $\mathrm{diag}(x)$ is the diagonal matrix whose diagonal are the elements of $x$, then we can write the Jacobian of $z \mapsto \begin{pmatrix}x \\ r \end{pmatrix}$ as

$$J = \begin{pmatrix}\mathrm{diag}(x) - x x^\top \\ r x^\top\end{pmatrix} = \begin{pmatrix}I_n - x \boldsymbol{1}_n^\top \\ r \boldsymbol{1}_n^\top \end{pmatrix} \mathrm{diag}(x),$$

where $\boldsymbol{x}_n$ is the $n$-vector of ones.
The Jacobian has a shape $(n+1,n)$.
Even though $\phi$ is bijective, the output of $\phi$ is represented with an extra coordinate, so that $J$ is non-square.
It is possible to construct $\phi$ as the composition of a bijective function and a semi-orthogonal transformation that lifts the output from an $n$-dimensional vector to an $(n+1)$-dimensional vector.

The Jacobian would then factorize as $J = Q H$, where $Q$ is a size $(n+1,n)$ semi-orthogonal matrix so that $Q^\top Q = I_n$.
Orthogonal transformations leave the volume unchanged, so the density correction depends only on the unknown $H$.
Then $\sqrt{|J^\top J|} = \sqrt{|H^\top Q^\top Q H|} = \sqrt{|H^\top H|} = |H|$ computes the density correction.

The matrix $P = I - x \boldsymbol{1}^\top_n$ is an orthogonal projection matrix, since if it operates twice on any vector $v$, the second action has no effect:

$$Pv = v - (\boldsymbol{1}^\top_n v) x$$
$$v - (\boldsymbol{1}^\top_n v) x - x (\boldsymbol{1}^\top_nv - (\boldsymbol{1}^\top_n v) (\boldsymbol{1}^\top_n x)) = v - (\boldsymbol{1}^\top_n v) x,$$

since $\boldsymbol{1}^\top_n x = \sum_{i=1}^n x_i = 1$.
As a result, $P^2 = P$ and $|P^\top P| = |P^2| = |P|$.
Then,

$$\sqrt{|J^\top J|} = \sqrt{|\mathrm{diag}(x)(I_n + (r^2 \boldsymbol{1}_n - x) \boldsymbol{1}_n^\top)\mathrm{diag}(x)|} = |\mathrm{diag}(x)|\sqrt{|I_n + (r^2 \boldsymbol{1}_n - x) \boldsymbol{1}_n^\top|}$$


$|\operatorname{diag}(x)| = \prod_{i=1}^n x_i = r^{-n} \exp (\sum_{i=1}^n z_i)$. 
Using Sylvester's determinant theorem, $|I_n + (r^2 \boldsymbol{1}_n - x) \boldsymbol{1}_n^\top| = 1 + \boldsymbol{1}_n^\top (r^2 \boldsymbol{1}_n - x) = 1 + n r^2 - \sum_{i=1}^n x_i = n r^2$, so the density correction is $\sqrt{n} (\sum_{i=1}^n e^{z_i})^{1-n} \exp(\sum_{i=1}^n z_i)$.

To keep the target distribution proper, we must select a prior distribution for $r$.
If we choose $r \sim \chi_n$, then the product of the correction and the density of the prior for $r$ is proportional to
$$\exp\left(\sum_{i=1}^n z_i - \frac{1}{2}(\sum_{i=1}^n e^{z_i})^2\right).$$

\vfill

\end{document}

\documentclass[twoside]{article}

\usepackage{aistats2021}
\usepackage{amssymb}

% If your paper is accepted, change the options for the package
% aistats2021 as follows:
%
%\usepackage[accepted]{aistats2021}
%
% This option will print headings for the title of your paper and
% headings for the authors names, plus a copyright note at the end of
% the first column of the first page.

% If you set papersize explicitly, activate the following three lines:
%\special{papersize = 8.5in, 11in}
%\setlength{\pdfpageheight}{11in}
%\setlength{\pdfpagewidth}{8.5in}

% If you use natbib package, activate the following three lines:
%\usepackage[round]{natbib}
%\renewcommand{\bibname}{References}
%\renewcommand{\bibsection}{\subsubsection*{\bibname}}

% If you use BibTeX in apalike style, activate the following line:
%\bibliographystyle{apalike}

\begin{document}

% If your paper is accepted and the title of your paper is very long,
% the style will print as headings an error message. Use the following
% command to supply a shorter title of your paper so that it can be
% used as headings.
%
%\runningtitle{I use this title instead because the last one was very long}

% If your paper is accepted and the number of authors is large, the
% style will print as headings an error message. Use the following
% command to supply a shorter version of the authors names so that
% they can be used as headings (for example, use only the surnames)
%
%\runningauthor{Surname 1, Surname 2, Surname 3, ...., Surname n}

% Supplementary material: To improve readability, you must use a single-column format for the supplementary material.
\onecolumn
\aistatstitle{Instructions for Paper Submissions to AISTATS 2021: \\
Supplementary Materials}

\section{FORMATTING INSTRUCTIONS}

To prepare a supplementary pdf file, we ask the authors to use \texttt{aistats2021.sty} as a style file and to follow the same formatting instructions as in the main paper.
The only difference is that the supplementary material must be in a \emph{single-column} format.
You can use \texttt{supplement.tex} in our starter pack as a starting point, or append the supplementary content to the main paper and split the final PDF into two separate files.

Note that reviewers are under no obligation to examine your supplementary material.

\section{Derivations of density corrections}

\subsection{Simplex softmax parameterization}

Let $\Delta^n$ indicate the unit $n$-simplex with $n$ positive elements that sum to 1 and $\Delta^n_-$ indicate the first $n-1$ elements, from which the final element can be uniquely determined.

We define the transformation $\phi: \mathbb{R}^{n-1} \to \Delta^n_-: y \mapsto x_-$, where $x=\begin{pmatrix}x_- \\ \frac{1}{r}\end{pmatrix} \in \Delta^n$, $x_i = \frac{1}{r} e^{y_i}$ for $1 \le i \le n-1$, and $r = 1 + \sum_{i=1}^{n-1} e^{y_i}$.

First we compute the scalar derivatives:
$$
\begin{aligned}
    \frac{\mathrm{d} r}{\mathrm{d} y_j} &= e^{y_j} = r x_j\\
    \frac{\mathrm{d} x_i}{\mathrm{d} y_j} &= \delta_{ij} \frac{1}{r} e^{y_i} - \frac{1}{r^2} e^{y_i} \frac{\mathrm{d} r}{\mathrm{d} y_j} = \delta_{ij} x_i - x_i x_j, \quad 1 \le i \le n-1
\end{aligned}
$$

where $\delta_{ij} = \begin{cases} 1 &\text{if } i = j \\ 0 &\text{otherwise}\end{cases}$ is the Kronecker delta.

If $\mathrm{diag}(x)$ is the diagonal matrix whose diagonal are the elements of $x$, then the Jacobian is

$$J = (I_{n-1} - x_- \boldsymbol{1}_{n-1}^\top) \operatorname{diag}(x_-),$$

where $\boldsymbol{1}_n$ is the $n$-vector of ones.

Using Sylvester's determinant theorem, $|I_{n-1} - x_- \boldsymbol{1}_{n-1}^\top| = 1 - \boldsymbol{1}_{n-1}^\top x_- = 1 - \sum_{i=1}^{n-1} x_i = x_n$, so

$$\mathrm{correction} = |J| = x_n \prod_{i=1}^{n-1} x_i = \prod_{i=1}^{n} x_i = \exp\left(\sum_{i=1}^{n-1} y_i\right) \left(1 + \sum_{i=1}^{n-1} e^{y_i}\right)^{-n}$$

\subsection{Simplex augmented softmax parameterization}

We define the transformation $\phi: \mathbb{R}^n \to \Delta^{n-1} \times \mathbb{R}_{>0}: y \mapsto (x_-, r)$,
where $r = \sum_{i=1}^n e^{y_i}$ and $x_i = \frac{1}{r} e^{y_i}$ for $1 \le i \le n-1$..

First we compute the scalar derivatives:

$$
\begin{aligned}
    \frac{\mathrm{d} r}{\mathrm{d} y_j} &= e^{y_j} = r x_j\\
    \frac{\mathrm{d} x_i}{\mathrm{d} y_j} &= \delta_{ij} \frac{1}{r} e^{y_i} - \frac{1}{r^2} e^{y_i} \frac{\mathrm{d} r}{\mathrm{d} y_j} = \delta_{ij} x_i - x_i x_j,
\end{aligned}
$$

which corresponds to the Jacobian

$$J = \begin{pmatrix}I_{n-1} - x_- \boldsymbol{1}_{n-1}^\top & -x_- \\ r \boldsymbol{1}_{n-1}^\top & r \end{pmatrix} \mathrm{diag}(x).$$

For invertible $A$, the determinant of the block matrix $\begin{pmatrix}A & B \\ C & D\end{pmatrix}$ is $|A| |D-CA^{-1}B|$.
A square matrix is invertible iff its determinant is non-yero.
From the previous section, $|I_{n-1} - x_- \boldsymbol{1}_{n-1}^\top| = x_n > 0$, so the determinant of the Jacobian is

$$|J| = x_n \left|r + r \boldsymbol{1}_{n-1}^\top (I_{n-1} - x_- \boldsymbol{1}_{n-1}^\top)^{-1} x_-\right| \prod_{i=1}^n x_i$$

Let $w = (I_{n-1} - x_- \boldsymbol{1}_{n-1}^\top)^{-1} x_-$. Then,

$$
\begin{aligned}
    w - x_- \sum_{i=1}^{n-1} w_i &= x_-\\
    w &= x_- \left(1 - \sum_{i=1}^{n-1} w_i\right)\\
    \sum_{i=1}^{n-1} w_i &= \sum_{i=1}^{n-1} \left( x_- (1 - \sum_{i=1}^{n-1} w_i) \right) = \left(\sum_{i=1}^{n-1} x_i \right) \left(1 - \sum_{i=1}^{n-1} w_i\right) = (1 - x_n)  \left(1 - \sum_{i=1}^{n-1} w_i\right)\\
    \sum_{i=1}^{n-1} w_i &= \frac{1 - x_n}{x_n} = \frac{1}{x_n} - 1\\
    w &= x_- \left(1 - \left(\frac{1}{x_n} - 1\right)\right) = \frac{1}{x_n} x_-
\end{aligned}
$$

Then

$$|J| = x_n r \left|1 + \frac{1}{x_n}\sum_{i=1}^{n-1} x_i\right| \prod_{i=1}^n x_i = r \prod_{i=1}^n x_i$$

To keep the target distribution proper, we must select a prior distribution $\pi(r)$ for $r$.
If we choose $r \sim \chi_n$, then the product of the correction and the density of the prior for $r$ is proportional to

$$\mathrm{correction} = \pi(r) |J| = r^n e^{-r^2/2} \prod_{i=1}^n x_i = \exp\left(\sum_{i=1}^n y_i - \frac{1}{2}\left(\sum_{i=1}^n e^{y_i}\right)^2\right).$$

Alternatively, if we choose $r \sim \mathrm{Gamma}(n, 1)$, then

$$\mathrm{correction} = \pi(r) |J| = r^n e^{-r} \prod_{i=1}^n x_i = \exp\left(\sum_{i=1}^n y_i - \sum_{i=1}^n e^{y_i}\right).$$

This latter correction is equivalent to the sampling procedure from the Dirichlet distribution with $\alpha_i=1$, where $z_i \sim \mathrm{Exponential}(1)$ and $y = \frac{z}{\sum_{i=1}^n z_i}$.

Both of these corrections can be captured with the generalization

$$\mathrm{correction} = \pi(r) |J| = r^n e^{-r^p/p} \prod_{i=1}^n x_i = \exp\left(\sum_{i=1}^n y_i - \frac{1}{p} \left(\sum_{i=1}^n e^{y_i}\right)^p\right),$$

for $p > 0$, which corresponds to $r \sim \text{Generalized-Gamma}(1, n, p)$.

\vfill

\end{document}
